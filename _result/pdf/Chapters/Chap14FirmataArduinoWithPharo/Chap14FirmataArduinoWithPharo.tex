% -*- mode: latex; -*- mustache tags:  
\documentclass[10pt,twoside,english]{_support/latex/sbabook/sbabook}
\let\wholebook=\relax

\usepackage{import}
\subimport{_support/latex/}{common.tex}

%=================================================================
% Debug packages for page layout and overfull lines
% Remove the showtrims document option before printing
\ifshowtrims
  \usepackage{showframe}
  \usepackage[color=magenta,width=5mm]{_support/latex/overcolored}
\fi


% =================================================================
\title{The Pillar Super Book Archetype}
\author{The Pillar team}
\series{Square Bracket tutorials}

\hypersetup{
  pdftitle = {The Pillar Super Book Archetype},
  pdfauthor = {The Pillar team},
  pdfkeywords = {project template, Pillar, Pharo, Smalltalk}
}


% =================================================================
\begin{document}

% Title page and colophon on verso
\maketitle
\pagestyle{titlingpage}
\thispagestyle{titlingpage} % \pagestyle does not work on the first one…

\cleartoverso
{\small

  Copyright 2017 by The Pillar team.

  The contents of this book are protected under the Creative Commons
  Attribution-ShareAlike 3.0 Unported license.

  You are \textbf{free}:
  \begin{itemize}
  \item to \textbf{Share}: to copy, distribute and transmit the work,
  \item to \textbf{Remix}: to adapt the work,
  \end{itemize}

  Under the following conditions:
  \begin{description}
  \item[Attribution.] You must attribute the work in the manner specified by the
    author or licensor (but not in any way that suggests that they endorse you
    or your use of the work).
  \item[Share Alike.] If you alter, transform, or build upon this work, you may
    distribute the resulting work only under the same, similar or a compatible
    license.
  \end{description}

  For any reuse or distribution, you must make clear to others the
  license terms of this work. The best way to do this is with a link to
  this web page: \\
  \url{http://creativecommons.org/licenses/by-sa/3.0/}

  Any of the above conditions can be waived if you get permission from
  the copyright holder. Nothing in this license impairs or restricts the
  author's moral rights.

  \begin{center}
    \includegraphics[width=0.2\textwidth]{_support/latex/sbabook/CreativeCommons-BY-SA.pdf}
  \end{center}

  Your fair dealing and other rights are in no way affected by the
  above. This is a human-readable summary of the Legal Code (the full
  license): \\
  \url{http://creativecommons.org/licenses/by-sa/3.0/legalcode}

  \vfill

  % Publication info would go here (publisher, ISBN, cover design…)
  Layout and typography based on the \textcode{sbabook} \LaTeX{} class by Damien
  Pollet.
}


\frontmatter
\pagestyle{plain}

\tableofcontents*
\clearpage\listoffigures

\mainmatter

\chapter{Lesson 14 –  Firmata implementation for the Pharo Programming Language.}
Firmata is a generic protocol for communicating with microcontrollers from software on a host computer. 
It is intended to work with any host computer software package. Right now there is a matching object in a number of languages. 
It is easy to add objects for other software to use this protocol. 
Basically, this firmware establishes a protocol for talking to the Arduino from the host software. 
The aim is to allow people to completely control the Arduino from software on the host computer.
\section{Installation}\subsection{Arduino: Installing Standard Firmata}
\begin{itemize}
\item Your first step should be to download the Arduino application from 
\end{itemize}

\textbf{\href{https://www.arduino.cc/en/Main/Software}{Arduino Website.}\footnote{\url{https://www.arduino.cc/en/Main/Software}}} 

\begin{itemize}
\item Be sure to choose the latest version and also the correct download for your computer and operating system.
\item Once the software has downloaded, you can install the application using the method appropriate for your system. 
\end{itemize}

\begin{displaycode}{plain}
  -For Mac OS X you will be downloading a ZIP file. Double-clicking on the ZIP should produce a single "Arduino" application file which you can then copy into your Applications folder.
  -For Windows, you should download the .EXE containing a full Windows installer. Double clicking on the .EXE should start the installation.
  -For Linux you will download a compressed TAR file. You can use the "tar" command to uncompress and unpack the application.
\end{displaycode}
\subsection{Pharo with Firmata}
To load latest version of Firmata you can evaluate the following expression in playground:

\begin{displaycode}{plain}
  Metacello new
  baseline: 'Firmata';
  repository: 'github://pharo-iot/Firmata';
  load
\end{displaycode}



\bibliographystyle{alpha}
\bibliography{book.bib}

% lulu requires an empty page at the end. That's why I'm using
% \backmatter here.
\backmatter

% Index would go here

\end{document}
